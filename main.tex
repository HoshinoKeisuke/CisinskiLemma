\documentclass[a4paper,  dvipsnames, 11pt]{amsart}
\usepackage{preamble}
\vspace{10ex}

\begin{document}
\maketitle
See \cite{Cis19} and the \href{https://mathoverflow.net/questions/467753/on-lemma-5-5-16-of-cisinskis-higher-categories-and-homotopical-algebra}{mathoverflow}.
The main objective is \Cref{prop:main}, which includes \cite[Lemma5.5.16]{Cis19}.
\begin{notation}
	We employ the following notations. 
	\begin{itemize}
		\item %
			Given any category $\one{C}$ and two objects $A,B\in\one{C}$,
			we write $\one{C}(A,B)$ for the homset.
		\item %
			$\Simp$ is the category of simplices.
			We write $\Delta^n$ for the $n$-simplex, which is also seen as a category. 
		\item %
			We write $\one{C}^\ra$ for the arrow category of $\one{C}$.
			We write $\Box$ for $\ra\times\ra$.
		\item %
			We write $\Set$ for the category of sets.
		\item %
			$\sSet$ is the category of simplicial sets; i.e., $\sSet=\left[\Simp^\op,\Set\right]$.
			Every set is seen as a simplicial set by the diagonal functor $\Set\arr\sSet$.
		\item %
			$\bisSet$ is the category of bisimplicial sets; i.e.,  $\bisSet=\left[\Simp^\op\times\Simp^\op,\Set\right]$
		\item %
			For each small category $\dgm{A}$, we write $\yo\colon\dgm{A}\arr\left[\dgm{A}^\op,\Set\right]$ for the yoneda embedding.
		\item %
			We write $-\boxtimes-\colon\sSet\times\sSet\arr\bisSet$ for the nerve functor of $\yo\times\yo\colon\Simp^\op\times\Simp^\op\arr\sSet\times\sSet$.
		\item %
			There are functors $-\rollipop-\colon\bisSet\times\sSet^\op\arr\sSet$ and $-\lollipop-\colon\sSet^\op\times\bisSet\arr\sSet$
			such that there exists the following family of bijections natural in $A,B\in\sSet^\op$ and $X\in\bisSet$.
			\begin{equation}
				\label{eq:thebiadjunct}
				\bisSet(A\boxtimes B,X)
				\cong
				\sSet(A,X\lollipop B)
				\cong
				\sSet(B,A\rollipop X)
			\end{equation}
		\item %
			We write $-\boxtimesL-\colon\sSet^\ra\times\sSet^\ra\arr\bisSet^\ra$ for the \textit{pushout product} or \textit{Leibniz product} with respect to $\boxtimes$;
			given two morphisms $f\colon A\arr B$ and $g\colon C\arr D$ in $\sSet$, the morphism $f\boxtimesL g$ is the canonical one of the form
			$A\boxtimes D \cup_{A\boxtimes C} B\boxtimes C\arr B\boxtimes D$ whose domain is the fibred coproduct of $f\boxtimes \id_{C}$ and $\id_{A}\boxtimes g$.
		\item %
			We write
			$-\rollipopL-\colon\sSet^{\ra,\op}\times\bisSet^\ra\arr\sSet^\ra$
			and
			$-\lollipopL-\colon\bisSet^\ra\times\sSet^{\ra,\op}\arr\sSet^\ra$
			for the \textit{pullback product} for $\rollipop$ and $\lollipop$ respectively.
			Therefore, given a morphism $f\colon A \arr B$ in $\sSet$ and a morphism $h\colon X\arr Y$ in $\bisSet$,
			$f\rollipopL h$ is the canonical morphism towards the fibred product of $f\rollipop \id_Y$ and $\id_A\rollipop h$.
		\item %
			Let $i$ and $k$ be morphisms.
			We write $i\pitchfork k$ if $i$ has left lifting property to $k$, or equivalently,
			$k$ has right lifting property to $i$.
		\item
			Given a class of morphisms $\I$,
			we write $\ell(\I)$ for the class of morphisms that have left lifting property to those in $\I$.
			Dually, we write $r(\I)$ for the class of morphisms that have right lifting property to those in $\I$.
		\item %
			Given two classes of morphisms $\I$ and $\K$ in $\sSet$,
			we obtain a class of morphisms $\I\boxtimesL\K=\{ i\boxtimesL k\,|\,i\in\I,\ k\in\K \}$.
		\item %
			We write $\I_\mono$ for the canonical cellular model for $\sSet$; i.e., the set of boundary inclusions.
		\item %
			We write $\I_\lft$ for the set of generating left anodyne extensions; i.e. the set of horn inclusions lifting initial objects.
		\item %
			We write $\I_\lftbi$ for $\I_\lft\boxtimesL\I_\mono\cup\I_\mono\boxtimesL\I_\lft$.
			The set
			$\I_\lftbi$ is the generator for left bi-anodyne extensions.
		\item %
			For each simplicial set $A$,
			we write $(\sSet/A)_\co$ for the slice category equipped with the homotopical structure defined for the covariant model structure,
		\item %
			For each bisimplicial set $X$,
			we write $(\bisSet/X)_\co$ for the slice category equipped with the homotopical structure for the bicovariant model structure.
		\item %
			We write $\omega$ for the least infinite ordinal seen as a category by its order.
		\item %
			We regard the horn $\Lambda^2_2$ as a subcategory of $\Delta^2$.
		\qedhere %
	\end{itemize}
\end{notation}
\section{On classes of morphisms in \texorpdfstring{$\bisSet$}{bisSet}}
\begin{lemma}
	\label{lem:pitchfork}
	For each morphisms $i,j$ in $\sSet$ and a morphism $h$ in $\bisSet$, the following are equivalent.
	\begin{itemize}
		\item %
			$i\boxtimesL j\pitchfork h$
		\item %
			$i\pitchfork h\lollipopL j$
		\item %
			$j\pitchfork i\rollipopL h$
	\end{itemize}
\end{lemma}
\begin{proof}
	Follows from \Cref{eq:thebiadjunct} and the universality of the pushout and the pullbacks defining
	$i\boxtimesL j$, $h\lollipopL j$, and $i\rollipopL h$.
\end{proof}
\begin{lemma}
	\label{lem:rIbtJ}
	Suppose we are given two small classes $\I$ and $\J$ of morphisms in $\sSet$, and a morphism $f$ in $\bisSet$.
	The following are equivalent.
	\begin{enumerate}
		\item %
			$h\in r(\I\boxtimesL\J)$.
		\item %
			For each $j\in\J$, $h\lollipopL j\in r(\I)$.
		\item %
			For each $i\in\I$, $i\rollipopL h\in r(\J)$.
		\item %
			For each $j\in\ell r(\J)$, $h\lollipopL j\in r(\I)$.
		\item %
			For each $i\in\ell r(\I)$, $i\rollipopL h\in r(\J)$.
	\end{enumerate}
\end{lemma}
\begin{proof}
	By \cref{lem:pitchfork}, the first three arguments are equivalent to the following.
	\[
		j \pitchfork i\rollipopL h
		\hspace{3ex}
		\text{for any $i\in\I$ and $j\in\J$.}
	\]
	By the small object argument,
	this is also equivalent to
	\[
		j \pitchfork i\rollipopL h
		\hspace{3ex}
		\text{for any $i\in\I$ and $j\in\ell r(\J)$,}
	\]
	which is equivalent to $iv)$.
	$v)$ is also equivalent to them in the same way.
\end{proof}
\begin{corollary}[{\cite[Lemma 5.5.6]{Cis19}}]
	\label{cor:biTrivfib}
	A morphism $q$ in $\bisSet$ is a trivial fibration
	if and only if
	$q\lollipop i$ is a trivial fibration
	for each monomorphism $i$ in $\sSet$.
\end{corollary}
\begin{proof}
	$\I_\mono\boxtimesL\I_\mono$ is a cellular model for $\bisSet$ (see \cite[Example 1.3.4 and 2.4.5]{Cis19}),
	and this is a direct consequence of \Cref{lem:rIbtJ}, where both $\I$ and $\J$ are $\I_\mono$.
\end{proof}
\begin{lemma}
	\label{lem:presliftTrivNaiveFibrations}
	$\dom\colon(\sSet/A)_\co\arr(\sSet)_\co$ preserves and lifts trivial fibrations and naive fibrations.
\end{lemma}
\begin{proof}
	Recall that $\dom$ preserves and lifts both monomorphisms and anodyne extensions.

	Let $f\colon S\arr T$ be a morphism in $\sSet/A$. We show that $f$ is a trivial/naive fibration if and only if $\dom(f)$ is a trivial/naive fibration.
	The if part is easily checked by considering that $\dom$ preserves monomorphisms/anodyne extensions.

	Let $i$ be a monomorphism/anodyne extension in $(\sSet)_\co$ and $(u,v)\colon i \arr \dom(f)$ be a morphism in $\sSet^\ra$.
	It suffices to show there is a diagonal filler for this square. 
	The morphisms towards $A$ from $f$ is followed by $(u,v)$, and this enables us to see $(u,v)$ as a
	square in $\sSet/A$. Moreover, its domain is a monomorphism/anodyne extension.
	Therefore, desired diagonal filler exists if $f$ is a trivial/naive fibration in $\sSet/A$.
\end{proof}
\begin{lemma}
	\label{lem:leftFibBWTrivFibs}
	A left fibration between trivial fibrations is a trivial fibration.
\end{lemma}
\begin{proof}
	Suppose we are given a commutative triangle
	\[
		\begin{tikzcd}[tri]
			A
			\ar[rr,"f"]
			\ar[rd,"h"',trivfib]
				&
					&
					B
					\ar[ld,"g",trivfib]
			\\
				&
				C
					&
		\end{tikzcd}
	\]
	in $\sSet$ such that $f$ is a left fibration (i.e., a naive fibration in $(\sSet)_\co$)
	and $g$ and $h$ are trivial fibrations. By \Cref{lem:presliftTrivNaiveFibrations}, this lifts to another triangle
	\[
		\begin{tikzcd}[tri]
			(A,h)
			\ar[rr,"f"]
			\ar[rd,"h"',trivfib]
				&
					&
					(B,g)
					\ar[ld,"g",trivfib]
			\\
				&
				(C,\id)
					&
		\end{tikzcd}
	\]
	in $(\sSet/C)_\co$ satisfying the same condition. Since $(A,h)$ and $(B,g)$ are fibrant, $f$ is a fibration. Therefore,
	the 2 out of 3 property for weak equivalences ensures that $f$ is a trivial fibration.
\end{proof}
\begin{proposition}
	\label{prop:bLeftbiFibvsTrivfib}
	Suppose we are given a morphism $q\colon X\arr Y$ in $\bisSet$ that belongs to $r(\I_\lft\boxtimesL\I_\mono)$
	and a monomorphism $i\colon K\arr[tail]L$ in $\sSet$.
	$q\lollipopL i$ is a trivial fibration if $q\lollipop K$ and $q\lollipop L$ are trivial fibrations.
\end{proposition}
\begin{proof}
	By \Cref{lem:rIbtJ}, $q\lollipopL i$ is a left fibration filling the following diagram.
	\[
		\begin{tikzcd}
			X\lollipop L
			\al[r,"X\lollipop i"]
			\al[d,"q\lollipop L"']
				&
				X\lollipop K
				\al[d,"x"]
				\al[rdd,bend left,"q\lollipop K"]
					&
			\\
			Y\lollipop L
			\al[r]
			\al[rrd,bend right]
				&
				\cdot
				\al[rd,"{q\lollipopL i}"description]
				\pullback[lu]
					&
			\\
				&
					&
					Y\lollipop K
		\end{tikzcd}
	\]
	The pullback square assures $x$ is also a trivial fibration, and this follows from \Cref{lem:leftFibBWTrivFibs}.
\end{proof}
\section{Insights from Reedy category theory}
\begin{observation}
	\label{obs:Leibniz}
	$\ra$ can be seen as a lattice so that there are
	the meet $\land\colon\Box\arr\ra$ and the join $\lor\colon\Box\arr\ra$.
	Given a category $\one{C}$ with pushout and pullback,
	we obtain $L_\land\colon\one{C}^\Box\arr\one{C}^\ra$ as the right Kan extension of the meet $\land$,
	while we have $L_\lor\colon\one{C}^\Box\arr\one{C}^\ra$ as the left Kan extension of the join $\lor$.

	On the other hand, suppose we are given a functor $\otimes\colon\one{A}\times\one{B}\arr\one{C}$.
	Consider the composite
	\[
		\ra\times\ra\times\one{A}^\ra\times\one{B}^\ra
		\arr"\text{evaluation}"[][6]
		\one{A}\times\one{B}
		\arr"\otimes"
		\one{C}\text{,}
	\]
	where the former is the canonical evaluation functor that sends $(i,j,X,Y)$ to $(X(i),Y(j))$.
	The currying of this composite gives a functor $\bar\otimes\colon\one{A}^\ra\times\one{B}^\ra\arr\one{C}^\Box$.

	Now we have two functors $L_\land\circ\bar\otimes$ and $L_\lor\circ\bar\otimes$ of the form $\one{A}^\ra\times\one{B}^\ra\arr\one{C}^\ra$.
	The first one is the \textit{pullback product} of $\otimes$, while the second one is the \textit{pushout product} or the \textit{Leibniz product} of $\otimes$.
\end{observation}
\begin{definition}
	\label{defn:SatMono}
	Suppose that we are given a cocomplete category $\one{C}$ and a class $S$ of objects in $\one{C}$.
	We say $S$ is \emph{saturated by monomorphisms}
	if for each diagram $F\colon\dI^\op\arr\one{C}$ satisfying the following conditions,
	the colimit of $F$ is in $S$;
	\begin{itemize}
		\item %
			$\dI$ is a discrete category, or either of
			$\omega^\op$ or $\Lambda^2_2$.
		\item %
			If $u\colon i\arr j$ in $\dI$ is of the form $n+1\arr n$ in $\omega^\op$ or $1\arr 2$ in $\Lambda^2_2$,
			then $F(u)$ is a monomorphism.
		\item %
			The image of $F$ is in $S$.
		\qedhere %
	\end{itemize}
\end{definition}
\begin{fact}[{\cite[Corollary 1.3.10]{Cis19}}]
	\label{fact:SatMono}
	A class of simplicial sets contains all simplicial sets
	if it is satureated by monomorphisms and contains representables.
\end{fact}
\begin{definition}
	\label{defn:ReedyTrivFib}
	Suppose that we are given
	\begin{itemize}
		\item %
			a complete category $\one{C}$,
		\item %
			a weak factorisation system $(\Cof,\TFib)$
			whose elements are called \textit{cofibrations} and \textit{trivial fibrations} respectively,
		\item %
			a small category $\dI$ that is a discrete category or either of $\omega^\op$ or $\Lambda^2_2$,
			and
		\item %
			two functors $F,G\colon\dI\arr\one{C}$.
	\end{itemize}
	A natural transformation $\alpha\colon F\arr[Rightarrow]G$
	is a
	\emph{Reedy trivial fibration}
	if it satisfies the followings.
	\begin{itemize}
		\item %
			For each object $i\in\dI$, $\alpha_i\colon F(i)\arr G(i)$ is a trivial fibration.
		\item %
			Let $u\colon i\arr j$ in $\dI$ be either of $n+1\arr n$ in $\omega^\op$ or $1\arr 2$ in $\Lambda^2_2$.
			Then the naturality square
			\[
				\begin{tikzcd}
					F(i)
					\ar[r,"F(u)"]
					\ar[d,"\alpha_i"']
						&
						F(j)
						\ar[d,"\alpha_j"]
					\\
					G(i)
					\ar[r,"G(u)"']
						&
						G(j)
				\end{tikzcd}
			\]
			is sent by $L_\land$ (see \Cref{obs:Leibniz}) to a trivial fibration.
		\qedhere %
	\end{itemize}
\end{definition}
\begin{proposition}
	\label{prop:ReedyLimit}
	The limit of a Reedy trivial fibration is a trivial fibration.
\end{proposition}
\begin{proof}
	Let us write $\Delta_\dI\colon\one{C}\arr\left[\dI,\one{C}\right]$
	for the diagonal functor, and write $\lim_\dI$ for its right adjoint.
	Suppose that we are given a cofibration $f\colon K\arr[tail]L$ in $\one{C}$
	and a Reedy trivial fibration $\alpha\colon F\arr G$.
	We show $f\pitchfork \lim_\dI(\alpha)$, which is equivalent to $\Delta_\dI(f)\pitchfork \alpha$ in $\left[\dI,\one{C}\right]$.
	To this end,
	let $(u,v)\colon\Delta_\dI(f)\arr\alpha$ be a morphism in $\left[\dI,\one{C}\right]^\ra$.
	It suffices to find the diagonal filler $k\colon\Delta_\dI(L)\arr F$ for this square.
	\begin{itemize}
		\item % discrete caty (product)
			When $\dI$ is a discrete category,
			take diagonal fillers $k_i\colon L\arr G(i)$ for
			each squares $(u_i,v_i)\colon f\arr\alpha_i$.
			Such fillers exist because $\alpha_i$ is a trivial fibration for each $i\in\dI$.
		\item % cocomposition
			Suppose $\dI=\omega^\op$.
			We construct the diagonal filler $(k_n)_{n\geq 0}$ by induction on $n$.
			\begin{itemize}
				\item[(n=0)] %
					$k_0$ is the diagonal filler for the square $(u_0,v_0)\colon f\arr\alpha_0$.
				\item[(n>0)] %
					Suppose we have constructed $k_{n-1}$ that fits into the following commutative diagram.
					\[
						\begin{tikzcd}
							K
							\ar[r,"u_n"]
							\ar[d,tail,"f"']
								&
								F(n)
								\ar[r]
								\ar[d]
									&
									F(n-1)
									\ar[d,"\alpha_{n-1}"]
							\\
							L
							\ar[rru,crossing over,"k_{n-1}"{description,near end}]
							\ar[r,"v_n"']
								&
								G(n)
								\ar[r]
									&
									G(n-1)
						\end{tikzcd}
					\]
					Since the square on the right is sent by $L_\land$ to a trivial fibration,
					we have another square whose right side is a tribial fibration;
					\[
						\begin{tikzcd}
							K
							\ar[r,"u_n"]
							\ar[d,"f"',tail]
								&
								F(n)
								\ar[d]
							\\
							K
							\ar[r,"\bar{k}_{n-1}"']
								&
								\cdot
						\end{tikzcd}
					\]
					where the bottom side is the morphism induced from $k_{n-1}$ and $v_n$ by the universality of the pullback defining the right side.
					We define $k_n$ as a filler for this square.
			\end{itemize}
		\item % pullback
			Suppose $\dI=\Lambda^2_2$. Define $(k_x)_{x=0,1,2}$ as follows.
			\begin{itemize}
				\item %
					$k_0$ is the filler for $(u_0,v_0)\colon f \arr \alpha_0$,
					and define $k_2$ as the composite
					\[
						L\arr"k_0"
						F(0)\arr F(2)
					\]
					where the latter is the image under $F$ of $0\arr 2$.
				\item %
					Now we have the following commutative diagram.
					\[
						\begin{tikzcd}
							K
							\ar[r,"u_1"]
							\ar[d,tail,"f"']
								&
								F(1)
								\ar[r]
								\ar[d]
									&
									F(2)
									\ar[d,"\alpha_{2}"]
							\\
							L
							\ar[rru,crossing over,"k_{2}"{description,near end}]
							\ar[r,"v_1"']
								&
								G(1)
								\ar[r]
									&
									G(2)
						\end{tikzcd}
					\]
					We define $k_1$ in the same way as the definition of $k_n$ above for $\dI=\omega^\op$.
				\qedhere
			\end{itemize}
	\end{itemize}
\end{proof}
\section{The main proposition}
\begin{definition}
	We say a morphism $i$ in $\bisSet$ 
	is a \emph{levelwise left anodyne extension} if
	$i\lollipop \Delta^n$ is a left anodyne extension for each $n\geq 0$.
\end{definition}
\begin{lemma}
	\label{lem:bLeftbiAnovsLLAno}
	Morphisms in $\ell r(\I_\lft\boxtimesL\I_\mono)$ are levelwise left anodyne extensions.
\end{lemma}
\begin{proof}
	Firstly, observe that the class of levelwise left anodyne extensions
	is described as
	\[
		\bigcup_{n\geq 0}
		((-)\lollipop {\Delta^n})^{-1}(\ell r(\I_\lft))
		\text{,}
	\]
	where $((-)\lollipop {\Delta^n})^{-1}$ means the inverse image of the function on morphisms
	induced from the functor $(-)\lollipop {\Delta^n}\colon\bisSet\arr\sSet$.
	Since $(-)\lollipop {\Delta^n}$ preserves colimits and saturated classes are closed under unions,
	this class is saturated.
	Therefore, it suffices to show that morphisms in $\I_\lft\boxtimesL\I_\mono$ are levelwise left anodyne extensions.

	Let $i\colon A\arr B\in\I_\lft$ and $j\colon C\arr D\in\I_\mono$.
	By applying $(-)\lollipop {\Delta^n}$ to the diagram defining $i\boxtimesL j$, we obtain the following diagram in $\sSet$.
	\[
		\begin{tikzcd}
			A\times C_n
			\ar[r,"\id\times j_n"]
			\ar[d,"i\times\id"']
				&
				A\times D_n
				\ar[d,"x"]
				\ar[rdd,bend left,"i\times\id"]
					&
			\\
			B\times C_n
			\ar[r]
			\ar[rrd,bend right]
				&
				\cdot
				\ar[rd,"{(i\boxtimesL j)\lollipop \Delta^n}"description]
				\pullback[lu]
					&
			\\
				&
					&
					B\times D_n
		\end{tikzcd}
	\]
	The two morphisms named $i\times\id$ are both left anodyne extensions since left anodyne extensions are closed under finite product (\cite[Proposition 3.4.3]{Cis19}).
	Since $(-)\lollipop\Delta^n$ preserves pushouts, the marked square is a pushout square, and hence $x$ is also a left anodyne extension.
	Moreover, since the outer square comprises monomorphisms, all morphisms in this diagram are monomorphisms \comment{citation}.
	Therefore, the resulting morphism $(i\boxtimesL j)\lollipop\Delta^n$ is a initial monomorphism, which is a left anodyne extension.
\end{proof}
\begin{lemma}
	\label{lem:MonoLevmono}
	A morphism $i$ in $\bisSet$ is a monomorphism if and only if $i\lollipop \Delta^n$ is a monomorphism for each $n\geq 0$.
\end{lemma}
\begin{proof}
	By the yoneda lemma and the fact that $(-)\lollipop\Delta^m$ is a right adjoint.
	In detail, the following are equivalent.
	\begin{itemize}
		\item %
			$i$ is a monomorphism.
		\item %
			$\bisSet(\Delta^n\boxtimes\Delta^m,i)$ is a monomorphism in $\Set$ for each $n,m\geq 0$.
		\item %
			$\sSet(\Delta^n,i\lollipop\Delta^m)$ is a monomorphism in $\Set$ for each $n,m\geq 0$.
		\item %
			$i\lollipop\Delta^m$ is a monomorphism in $\sSet$ for each $m\geq 0$.
		\qedhere %
	\end{itemize}
\end{proof}
\begin{proposition}
	\label{prop:main}
	Levelwise left anodyne extensions are left bi-anodyne extensions.
\end{proposition}
\begin{proof}
	Let $i\colon X\arr Y$ be a levelwise left anodyne extension.
	\Cref{lem:MonoLevmono} shows $i$ is a monomorphism.
	By the small object argument, we have a weak factorisation system $(\ell r (\I_\lft\boxtimesL\I_\mono),r(\I_\lft\boxtimesL\I_\mono))$,
	and let us write
	\[
		\begin{tikzcd}[tri]
			X
			\ar[rr,"i"]
			\ar[rd,"j"']
				&
					&
					Y
			\\
				&
				Z
				\ar[ru,"q"']
					&
		\end{tikzcd}
	\]
	for the factorisation of $i$ with respect to this system.
	Since $j$ is in particular a left bi-anodyne extension and $i$ is a monomorphism, it suffices to show
	$q$ is a trivial fibration.
	Moreover, \Cref{cor:biTrivfib,prop:bLeftbiFibvsTrivfib} ensures that it suffices to show
	$q\lollipop K$ is a trivial fibration for each simplicial set $K$.

	Consider the case when $K=\Delta^n$ for some $n\geq0$.
	\Cref{lem:rIbtJ} shows $q\lollipop\Delta^n$ is a left fibration and
	\Cref{lem:bLeftbiAnovsLLAno} shows $j\lollipop\Delta^n$ is a left anodyne extension.
	Since $i\lollipop\Delta^n$ is assumed to be a left anodyne extension,
	$q\lollipop\Delta^n$ is a initial left fibration, which is a trivial fibration.

	By Corollary 1.3.10 of \cite{Cis19} (see \Cref{fact:SatMono}), it suffices to show that
	the class
	\[
		S=\{K\in\sSet\,|\,q\lollipop K\text{ is a trivial fibration }\}
	\]
	is saturated by monomorphisms.
	Let $F\colon\dI^\op\arr\sSet$ be a diagram satisfying the property described in \Cref{defn:SatMono}.
	Since $Z\lollipop -$ sends colimits to limits for any $Z\in\bisSet$,
	by virtue of \Cref{prop:ReedyLimit},
	it suffices to show that the natural transformation
	\begin{equation}
		\label{eq:DesReedytfib}
		q\lollipop F(-)\colon Z\lollipop F(-)\arr[Rightarrow]Y\lollipop F(-)\colon \dI\arr\sSet
	\end{equation}
	is a Reedy trivial fibration.
	Now considering \Cref{obs:Leibniz}, one can check that \Cref{prop:bLeftbiFibvsTrivfib} shows that
	the naturality square for the transformation $q\lollipop -$ at a monomorphism
	is sent by $L_\land$ to a trivial fibration.
	This shows \Cref{eq:DesReedytfib} is a Reedy trivial fibration, which completes the proof.
\end{proof}

\bibliographystyle{halpha-abbrv}
\bibliography{bibliography}

\end{document}

\begin{corollary}
	Morphisms in $r(\I_\lft\boxtimesL\I_\mono)$ are levelwise left fibrations.
\end{corollary}
\begin{proof}
	Suppose we are given a simplicial set $K$ and a morphism $q$ in $\bisSet$.
	Let us write $!_K\colon\emptyset\arr K$ for the unique morphism.
	Then $q\lollipopL !_K$ is isomorphic to $q\lollipop K$ and $!_K$ is in $\ell r(\I_\mono)$ since it is a monomorphism.
	By applying \Cref{lem:rIbtJ}, we conclude that $q\lollipop K\cong i\lollipopL !_K$ is in $r(\I_\lft)$ if $q$ is in $r(\I_\lft\boxtimesL\I_\mono)$.
\end{proof}
